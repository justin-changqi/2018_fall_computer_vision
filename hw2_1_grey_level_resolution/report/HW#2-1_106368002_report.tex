\documentclass[a4paper, 11pt]{article}
    \usepackage{comment} % enables the use of multi-line comments (\ifx \fi) 
    \usepackage{lipsum} %This package just generates Lorem Ipsum filler text. 
    \usepackage{fullpage} % changes the margin
    \usepackage{CJKutf8}
    \usepackage{enumitem}
    \usepackage{titlesec}
    \usepackage{graphicx}     % for figure
    \usepackage{subcaption}   % for figure
    \usepackage[export]{adjustbox}
\renewcommand{\theenumi}{\alph{enumi}}

\titlespacing*{\section}
  {0pt}{0.5\baselineskip}{1\baselineskip}

\titlespacing*{\subsection}
  {0pt}{0.1\baselineskip}{0.1\baselineskip}

\begin{document}
%Header-Make sure you update this information!!!!
\noindent
\begin{center}
  \large\textbf{2018 Fall Advance Digital Image Processing Homework \#2-1} \\
\end{center}
\begin{CJK}{UTF8}{bsmi}
\normalsize EE 245765 \hfill \textbf{106368002 張昌祺 Justin, Chang-Qi Zhang} \\
Advisor: 電子所 高立人 \hfill justin840727@gmail.com \\
\null\hfill Due Date: 13:00pm, Oct 2 2018 \\
\end{CJK}

\section*{Problem 1 Grey-level resolution with C++}
\begin{enumerate}[label=\alph*.]
  \item Using C/C++ to quantize the gray-level resolution of lena\_256.raw and 
  baboon\_256.raw from 8 bits to 1 bit. Show the results of these quantize images 
  and explain the difference between each result image. (Figure, 15\%; Discussion, 10\%)
  \subsection*{Ans}
  Firstly we take a look the result images which are generated by my program for both 
  Lena and baboon grey-level resolution from 8 bits to 1 bit.
  \begin{figure}[h]
    \centering
    \begin{subfigure}[b]{0.2\linewidth}
      \includegraphics[width=\linewidth]{{"../result_img_2_1/lena 8 bits"}.png}
       \caption{Lena 8 bit.}
    \end{subfigure}
    \begin{subfigure}[b]{0.2\linewidth}
      \includegraphics[width=\linewidth]{{"../result_img_2_1/lena 7 bits"}.png}
      \caption{Lena 7 bits.}
    \end{subfigure}
    \begin{subfigure}[b]{0.2\linewidth}
      \includegraphics[width=\linewidth]{{"../result_img_2_1/lena 6 bits"}.png}
      \caption{Lena 6 bits.}
    \end{subfigure}
    \begin{subfigure}[b]{0.2\linewidth}
      \includegraphics[width=\linewidth]{{"../result_img_2_1/lena 5 bits"}.png}
      \caption{Lena 5 bits.}
    \end{subfigure}
    \begin{subfigure}[b]{0.2\linewidth}
      \includegraphics[width=\linewidth]{{"../result_img_2_1/lena 4 bits"}.png}
       \caption{Lena 4 bits.}
    \end{subfigure}
    \begin{subfigure}[b]{0.2\linewidth}
      \includegraphics[width=\linewidth]{{"../result_img_2_1/lena 3 bits"}.png}
      \caption{Lena 3 bits.}
    \end{subfigure}
    \begin{subfigure}[b]{0.2\linewidth}
      \includegraphics[width=\linewidth]{{"../result_img_2_1/lena 2 bits"}.png}
      \caption{Lena 2 bits.}
    \end{subfigure}
    \begin{subfigure}[b]{0.2\linewidth}
      \includegraphics[width=\linewidth]{{"../result_img_2_1/lena 1 bits"}.png}
      \caption{Lena 1 bits.}
    \end{subfigure}
    \caption{lena\_256.raw grey-level resolution from 8 bits to 1 bit.}
    \label{fig:lena8to1bit}
  \end{figure}
  \begin{figure}[h]
    \centering
    \begin{subfigure}[b]{0.2\linewidth}
      \includegraphics[width=\linewidth]{{"../result_img_2_1/baboon 8 bits"}.png}
       \caption{Baboon 8 bit.}
    \end{subfigure}
    \begin{subfigure}[b]{0.2\linewidth}
      \includegraphics[width=\linewidth]{{"../result_img_2_1/baboon 7 bits"}.png}
      \caption{Baboon 7 bits.}
    \end{subfigure}
    \begin{subfigure}[b]{0.2\linewidth}
      \includegraphics[width=\linewidth]{{"../result_img_2_1/baboon 6 bits"}.png}
      \caption{Baboon 6 bits.}
    \end{subfigure}
    \begin{subfigure}[b]{0.2\linewidth}
      \includegraphics[width=\linewidth]{{"../result_img_2_1/baboon 5 bits"}.png}
      \caption{Baboon 5 bits.}
    \end{subfigure}
    \label{fig:baboon8to1bit}
  \end{figure}
  \begin{figure}[h]
    \ContinuedFloat
    \centering
    \begin{subfigure}[b]{0.2\linewidth}
      \includegraphics[width=\linewidth]{{"../result_img_2_1/baboon 4 bits"}.png}
       \caption{Lena 4 bit.}
    \end{subfigure}
    \begin{subfigure}[b]{0.2\linewidth}
      \includegraphics[width=\linewidth]{{"../result_img_2_1/baboon 3 bits"}.png}
      \caption{Lena 3 bits.}
    \end{subfigure}
    \begin{subfigure}[b]{0.2\linewidth}
      \includegraphics[width=\linewidth]{{"../result_img_2_1/baboon 2 bits"}.png}
      \caption{Lena 2 bits.}
    \end{subfigure}
    \begin{subfigure}[b]{0.2\linewidth}
      \includegraphics[width=\linewidth]{{"../result_img_2_1/baboon 1 bits"}.png}
      \caption{Lena 1 bits.}
    \end{subfigure}
    \caption{baboon\_256.raw grey-level resolution from 8 bits to 1 bit.}
    \label{fig:baboon8to1bit}
  \end{figure}
  \newpage
  In this section, we compare \textbf{False Contouring} between Lena 
  (Figure~\ref{fig:lena8to1bit}) and Baboon (Figure~\ref{fig:baboon8to1bit}) images. 
  In Lena's case, when the grey-level resolution down to 3 bits. The figure shows obvious 
  False contouring. In Baboon case, the false contouring effect happens in 2 bits grey-level 
  resolution. Then we know false contouring might happen in different bit number in different
  detail images. For low detail image like Lena, we need to represent the image with more 
  bits than high detail baboon image. The results for this problem is matching the 
  Isopreference Curve theory.
  \item Calculate the corresponding with MSE (Mean Square Error, study yourself) and PSNR value.
  (Discussion, 10\%)
  \subsection*{Ans}
  For calculate the MES (Mean square error) for the images, we use Equation~(\ref{eqr:mse}) 
  \cite{MSE-wiki}.
  \begin{equation}
    MSE=\frac{1}{n}\sum_{i=1}^{n}(I_{i}-\hat{I}_{i})^2
    \label{eqr:mse}
  \end{equation}
  The total number of pixels \textbf{n} is define as images $width \times height$. And 
  we sum up the square of pixel difference between original image \textbf{$I_{i}$} and resample
  image \textbf{$\hat{I_{i}}$}.

  \vspace{1em}
  The equation for PSNR (Peak signal-to-noise ratio) value shows in Equation~(\ref{eqr:psnr}) below.
  \begin{equation}
    PSNR=10\cdot \log_{10}(\frac{MAX_I^2}{MSE})
    \label{eqr:psnr}
  \end{equation}
  Here, the $MAX_I$ is the maximux pixel value of the image. For example, In 8 bits case, the 
  $MAX_I=2^8-1=255$. MSE is same value that is calculated by Equation~(\ref{eqr:mse}) \cite{PSNR-wiki}.

  \vspace{1em}
  The execute results for MSE and PSNR of different grey-level resolution are show on 
  Figure~\ref{fig:hw2_1b}.
  \begin{figure}[h]
    \centering
    \adjincludegraphics[width=.5\textwidth, trim={0 {0.76\height} 0 0},clip]{{"img_src/hw2_1b"}.png}
  \end{figure}
  \begin{figure}[h]
    \centering
    \adjincludegraphics[width=.5\textwidth, trim={0 0 0 {0.24\height}},clip]{{"img_src/hw2_1b"}.png}
    \caption{MSE and PSNR results for different grey-level resolution.}
    \label{fig:hw2_1b}
  \end{figure}

  The result shows that if we represent the image with more bits. The MSE will be lower and 
  the PSNR value is higher. When there is no error between two images ($MSE=0$), the PSNR will 
  become infinity high.
\end{enumerate}
\subsection*{Source code for Problem 1}
\it{hw2\_1\_grey\_level\_resolution.hpp}
\begin{figure}[h]
  \centering
  \adjincludegraphics[width=\textwidth, trim={0 {0.01\height} 0 0},clip]{{"img_src/hw2_1hpp"}.png}
\end{figure}
\newpage
\it{hw2\_1\_grey\_level\_resolution.cpp}
\begin{figure}[h!]
  \centering
  \adjincludegraphics[width=\textwidth, trim={0 {0.576\height} 0 0},clip]{{"img_src/hw2_1cpp"}.png}
\end{figure}
\begin{figure}[h!]
  \centering
  \adjincludegraphics[width=\textwidth, trim={0 {0.155\height} 0 {0.424\height}},clip]{{"img_src/hw2_1cpp"}.png}
\end{figure}
\begin{figure}[h!]
  \centering
  \adjincludegraphics[width=\textwidth, trim={0 0 0 {0.845\height}},clip]{{"img_src/hw2_1cpp"}.png}
\end{figure}
\newline
\begin{thebibliography}{9}
  \bibitem{MSE-wiki} Wikipedia. \emph{Mean squared error}[online]. \\
  Available from World Wide Web: (https://en.wikipedia.org/wiki/Mean\_squared\_error).
  \bibitem{PSNR-wiki} Wikipedia. \emph{Peak signal-to-noise ratio}[online].\\
  Available from World Wide Web: \\
  (https://en.wikipedia.org/wiki/Peak\_signal-to-noise\_ratio).
\end{thebibliography}
\end{document}