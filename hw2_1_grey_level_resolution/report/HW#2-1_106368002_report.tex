\documentclass[a4paper, 11pt]{article}
    \usepackage{comment} % enables the use of multi-line comments (\ifx \fi) 
    \usepackage{lipsum} %This package just generates Lorem Ipsum filler text. 
    \usepackage{fullpage} % changes the margin
    \usepackage{CJKutf8}
    \usepackage{enumitem}
    \usepackage{titlesec}
    \usepackage{graphicx}     % for figure
    \usepackage{subcaption}   % for figure
\renewcommand{\theenumi}{\alph{enumi}}

\titlespacing*{\section}
  {0pt}{0.5\baselineskip}{1\baselineskip}

\titlespacing*{\subsection}
  {0pt}{0.1\baselineskip}{0.1\baselineskip}

\begin{document}
%Header-Make sure you update this information!!!!
\noindent
\begin{center}
  \large\textbf{2018 Fall Advance Digital Image Processing Homework \#2-1} \\
\end{center}
\begin{CJK}{UTF8}{bsmi}
\normalsize EE 245765 \hfill \textbf{106368002 張昌祺 Justin, Chang-Qi Zhang} \\
Advisor: 電子所 高立人 \hfill justin840727@gmail.com \\
\null\hfill Due Date: 13:00pm, Oct 2 2018 \\
\end{CJK}

\section*{Problem 1 Grey-level resolution with C++}
\begin{enumerate}[label=\alph*]
  \item Using C/C++ to quantize the gray-level resolution of lena\_256.raw and 
  baboon\_256.raw from 8 bits to 1 bit. Show the results of these quantize images 
  and explain the difference between each result image. (Figure, 15\%; Discussion, 10\%)
  \subsection*{Ans}
  Firstly we take a look the result images which are generated by my program for both 
  Lena and baboon grey-level resolution from 8 bits to 1 bit.
  \begin{figure}[h]
    \centering
    \begin{subfigure}[b]{0.2\linewidth}
      \includegraphics[width=\linewidth]{{"../result_img_2_1/lena 8 bits"}.png}
       \caption{Lena 8 bit.}
    \end{subfigure}
    \begin{subfigure}[b]{0.2\linewidth}
      \includegraphics[width=\linewidth]{{"../result_img_2_1/lena 7 bits"}.png}
      \caption{Lena 7 bits.}
    \end{subfigure}
    \begin{subfigure}[b]{0.2\linewidth}
      \includegraphics[width=\linewidth]{{"../result_img_2_1/lena 6 bits"}.png}
      \caption{Lena 6 bits.}
    \end{subfigure}
    \begin{subfigure}[b]{0.2\linewidth}
      \includegraphics[width=\linewidth]{{"../result_img_2_1/lena 5 bits"}.png}
      \caption{Lena 5 bits.}
    \end{subfigure}
    \begin{subfigure}[b]{0.2\linewidth}
      \includegraphics[width=\linewidth]{{"../result_img_2_1/lena 4 bits"}.png}
       \caption{Lena 4 bits.}
    \end{subfigure}
    \begin{subfigure}[b]{0.2\linewidth}
      \includegraphics[width=\linewidth]{{"../result_img_2_1/lena 3 bits"}.png}
      \caption{Lena 3 bits.}
    \end{subfigure}
    \begin{subfigure}[b]{0.2\linewidth}
      \includegraphics[width=\linewidth]{{"../result_img_2_1/lena 2 bits"}.png}
      \caption{Lena 2 bits.}
    \end{subfigure}
    \begin{subfigure}[b]{0.2\linewidth}
      \includegraphics[width=\linewidth]{{"../result_img_2_1/lena 1 bits"}.png}
      \caption{Lena 1 bits.}
    \end{subfigure}
    \caption{lena\_256.raw grey-level resolution from 8 bits to 1 bit.}
    \label{fig:lena8to1bit}
  \end{figure}
  \begin{figure}[h]
    \centering
    \begin{subfigure}[b]{0.2\linewidth}
      \includegraphics[width=\linewidth]{{"../result_img_2_1/baboon 8 bits"}.png}
       \caption{Baboon 8 bit.}
    \end{subfigure}
    \begin{subfigure}[b]{0.2\linewidth}
      \includegraphics[width=\linewidth]{{"../result_img_2_1/baboon 7 bits"}.png}
      \caption{Baboon 7 bits.}
    \end{subfigure}
    \begin{subfigure}[b]{0.2\linewidth}
      \includegraphics[width=\linewidth]{{"../result_img_2_1/baboon 6 bits"}.png}
      \caption{Baboon 6 bits.}
    \end{subfigure}
    \begin{subfigure}[b]{0.2\linewidth}
      \includegraphics[width=\linewidth]{{"../result_img_2_1/baboon 5 bits"}.png}
      \caption{Baboon 5 bits.}
    \end{subfigure}
    \label{fig:baboon8to1bit_1}
  \end{figure}
  \begin{figure}[h]
    \ContinuedFloat
    \centering
    \begin{subfigure}[b]{0.2\linewidth}
      \includegraphics[width=\linewidth]{{"../result_img_2_1/baboon 4 bits"}.png}
       \caption{Lena 4 bit.}
    \end{subfigure}
    \begin{subfigure}[b]{0.2\linewidth}
      \includegraphics[width=\linewidth]{{"../result_img_2_1/baboon 3 bits"}.png}
      \caption{Lena 3 bits.}
    \end{subfigure}
    \begin{subfigure}[b]{0.2\linewidth}
      \includegraphics[width=\linewidth]{{"../result_img_2_1/baboon 2 bits"}.png}
      \caption{Lena 2 bits.}
    \end{subfigure}
    \begin{subfigure}[b]{0.2\linewidth}
      \includegraphics[width=\linewidth]{{"../result_img_2_1/baboon 1 bits"}.png}
      \caption{Lena 1 bits.}
    \end{subfigure}
    \caption{baboon\_256.raw grey-level resolution from 8 bits to 1 bit.}
    \label{fig:baboon8to1bit_2}
  \end{figure}
  \newpage
  \item Calculate the corresponding with MSE (Mean Square Error, study yourself) and PSNR value.
  (Discussion, 10\%)
  \subsection*{Ans}
  \subsection*{Source code}
\end{enumerate}


% \section*{Investigation/Research}
% \lipsum[2]
    
\begin{thebibliography}{9}
	\bibitem{Robotics} Fred G. Martin \emph{Robotics Explorations: A Hands-On Introduction to Engineering}. New Jersey: Prentice Hall.
	\bibitem{Flueck}  Flueck, Alexander J. 2005. \emph{ECE 100}[online]. Chicago: Illinois Institute of Technology, Electrical and Computer Engineering Department, 2005 [cited 30
	August 2005]. Available from World Wide Web: (http://www.ece.iit.edu/~flueck/ece100).
\end{thebibliography}
    
\end{document}