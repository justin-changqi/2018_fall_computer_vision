\documentclass[a4paper, 11pt]{article}
    \usepackage{comment} % enables the use of multi-line comments (\ifx \fi) 
    \usepackage{lipsum} %This package just generates Lorem Ipsum filler text. 
    \usepackage{fullpage} % changes the margin
    \usepackage{CJKutf8}
    \usepackage{enumitem}
    \usepackage{titlesec}
    \usepackage[english]{babel}
    \usepackage{blindtext}
    \usepackage{graphicx}     % for figure
    \usepackage{subcaption}   % for figure
    \usepackage[export]{adjustbox}
    \usepackage[most]{tcolorbox}
    \usepackage{xcolor}
    \usepackage{listings} % show code
\renewcommand{\theenumi}{\alph{enumi}}

\definecolor{codegreen}{rgb}{0,0.6,0}
\definecolor{codegray}{rgb}{0.5,0.5,0.5}
\definecolor{codepurple}{rgb}{0.58,0,0.82}
\definecolor{backcolour}{rgb}{0.95,0.95,0.92}

\lstdefinestyle{mystyle}{
    backgroundcolor=\color{backcolour},   
    commentstyle=\color{codegreen},
    keywordstyle=\color{magenta},
    numberstyle=\tiny\color{codegray},
    stringstyle=\color{codepurple},
    basicstyle=\footnotesize,
    breakatwhitespace=false,         
    breaklines=true,                 
    captionpos=b,                    
    keepspaces=true,                 
    numbers=left,                    
    numbersep=5pt,                  
    showspaces=false,                
    showstringspaces=false,
    showtabs=false,                  
    tabsize=2
}

\lstset{style=mystyle}

\titlespacing*{\section}
  {0pt}{0.5\baselineskip}{1\baselineskip}

\titlespacing*{\subsection}
  {0pt}{0.1\baselineskip}{0.1\baselineskip}

\begin{document}
%Header-Make sure you update this information!!!!
\noindent
\begin{center}
  \large\textbf{2018 Fall Advance Digital Image Processing Homework \#2-2} \\
\end{center}
\begin{CJK}{UTF8}{bsmi}
\normalsize EE 245765 \hfill \textbf{106368002 張昌祺 Justin, Chang-Qi Zhang} \\
Advisor: 電子所 高立人 \hfill justin840727@gmail.com \\
\null\hfill Due Date: 13:00pm, Oct 9 2018 \\
\end{CJK}

\section*{Problem 2 Zooming and Shrinking (C/C++)}
\begin{enumerate}[label=\alph*.]
  \item 
    Zooming the image with ratio 2:1 raw-column replication. Compare the output with
    lena512.raw. (Figure, 10\%; Discussion, 5\%)
    \subsection*{Ans}
    In Figure~\ref{fig:row-col-replication}, it is very clear to describe how
    row-col-replication works to achieve rooming image. Scale step, we multiply row index an 
    column index with scale factor (2 in this case). Row and column replication are simply 
    duplicate the row $i$ and column $j$ to row $i+1$ and column $j+1$.

    \begin{figure}[h]
      \centering
      \includegraphics[width=.9\textwidth]{{"img_src/row-col replication"}.png}
      \caption{Concept of row-col replication.}
      \label{fig:row-col-replication}
    \end{figure}
    Figure~\ref{fig:row-col-replication-Lena} shows the original Lena 512
    image(result\_img/lena\_512.png) and the result(result\_img/2-a zooming 
    lena row-col replication.png) of row-col replication from Lena 256 image. 
    Then you can see there is checkerboard effect on row-col replication result.
    \begin{figure}[h]
      \centering
      \begin{subfigure}[b]{0.4\linewidth}
        \includegraphics[width=\textwidth]{{"../result_img/lena_512"}.png}
        \caption{Lena 512 original.}
      \end{subfigure}
      \hspace{5em}
      \begin{subfigure}[b]{0.4\linewidth}
        \includegraphics[width=\textwidth]{{"../result_img/2-a zooming lena row-col replication"}.png}
        \caption{Row-col replication from Lena 256.}
      \end{subfigure}
      \caption{Lena 512 and Lena 256 Row-col replication.}
      \label{fig:row-col-replication-Lena}
    \end{figure}
    \newpage
    I calculated MSE and PSNR between Lena 512 and col-row replication. The running result
    as Figure~\ref{fig:row-col-replication-mse-psnr}. The data is loss a lot here.
    The typical PSNR value for video compression are between 30 to 50 dB. 
    \begin{figure}[h]
      \centering
      \includegraphics[width=.5\textwidth]{{"img_src/2-a-mse-psnr"}.png}
      \caption{MSE and PSNR result.}
      \label{fig:row-col-replication-mse-psnr}
    \end{figure}
  \item 
    Shrinking the image with ratio 1:2 raw-column deletion. Check your result with or without
    blurring (using Xnview) your input image before shrinking. (Figure, 10\%; Discussion, 5\%)
    \subsection*{Ans}
    Raw-column deletion is a simple method to shrinking image. The difference between 
    Figure~\ref{fig:row-col-deletion-Lena} and Figure~\ref{fig:row-col-deletion-blur-Lena}
    is that Figure~\ref{fig:row-col-deletion-Lena} direct compute rosw-col deletion and 
    Figure~\ref{fig:row-col-deletion-blur-Lena} use gaussian blur first then compute ros-col deletion.
    Row-col deletion is a sampling method so it is very easy to get the aliasing effect, if the input image
    is a high detail image. For solving aliasing effect we can make the image blur before we apply row-col
    deletion.
    \begin{figure}[h]
      \centering
      \begin{subfigure}[b]{0.3\linewidth}
        \includegraphics[width=\textwidth]{{"../result_img/lena_512"}.png}
        \caption{Lena 256 original.}
      \end{subfigure}
      \hspace{5em}
      \begin{subfigure}[b]{0.15\linewidth}
        \includegraphics[width=\textwidth]{{"../result_img/2-b-1 shrinking lena row-col deletion"}.png}
        \caption{Row-col deletion from Lena 256.}
      \end{subfigure}
      \caption{Results of Lena 256 Row-col deletion.}
      \label{fig:row-col-deletion-Lena}
    \end{figure}
    % \newpage
    \begin{figure}[h]
      \centering      
      \begin{subfigure}[b]{0.3\linewidth}
        \includegraphics[width=\textwidth]{{"../result_img/2-b-2 lena 256 blur"}.png}
        \caption{Lena 256 gaussian blur.}
      \end{subfigure}
      \hspace{5em}
      \begin{subfigure}[b]{0.15\linewidth}
        \includegraphics[width=\textwidth]{{"../result_img/2-b-3 lena 256 blur row-col deletion"}.png}
        \caption{Row-col deletion from Lena 256 gaussian blur.}
      \end{subfigure}
      \caption{Results of Lena 256 gaussian blur Row-col deletion.}
      \label{fig:row-col-deletion-blur-Lena}
    \end{figure}
    \newpage
  \item 
    Zooming the image with ratio 2.3 using both nearest-neighboring and bilinear interpolation.
    Discuss the difference in the output images. (Figure, 10\%; Discussion, 5\%)
    \subsection*{Ans}
    On results of those two method, we can see in result of \textit{nearest neighboring} got
    obvious checkerboard effect but it does not happen on  result of 
    \textit{bilinear interpolation}.
    \begin{figure}[h]
      \centering
      \begin{subfigure}[b]{0.4\linewidth}
        \includegraphics[width=\textwidth]{{"../result_img/2-c-1 zooming lena nearest neighboring"}.png}
        \caption{Zooming with nearest-neighboring}
      \end{subfigure}
      \hspace{1em}
      \begin{subfigure}[b]{0.4\linewidth}
        \includegraphics[width=\textwidth]{{"../result_img/2-c-2 zooming lena bilinear interpolation"}.png}
        \caption{Zooming with bilinear interpolation}
      \end{subfigure}
      \caption{Zooming results for \textit{nearest-neighboring} and \textit{bilinear interpolation}}
      \label{fig:row-col-rooming-high-Lena}
    \end{figure}
\end{enumerate}
\subsection*{Source code for Problem 2}
\textit{hw2\_2\_rooming\_shrinking.hpp}
\lstinputlisting[language=C++]{../include/hw2_2_rooming_shrinking.hpp}
\textit{hw2\_2\_rooming\_shrinking.cpp}
\lstinputlisting[language=C++]{../src/hw2_2_rooming_shrinking.cpp}

\section*{Problem 3 Isopreference test (C/C++)}
  Experiment the isopreference test on lena\_256.raw and baboon\_256.raw images with your
  programs developed in Problems 1 \& 2. Do your experiments and observations match the textbook
  description? Discuss it. (Discussion, 20\%)
  \subsection*{Ans}
  In textbook, it just mention about the isopreference with different gray-level resolution. 
  There is no any section which discussion the relationship between rooming, shrinking and 
  gray-level resolution. The experiments in this section is try to zooming and shrinking 
  the high detail(Baboon) and low detail(Lena) image in different gray-level resolution. 
% \begin{thebibliography}{9}
%   \bibitem{MSE-wiki} Wikipedia. \emph{Mean squared error}[online]. \\
%   Available from World Wide Web: (https://en.wikipedia.org/wiki/Mean\_squared\_error).
%   \bibitem{PSNR-wiki} Wikipedia. \emph{Peak signal-to-noise ratio}[online].\\
%   Available from World Wide Web: \\
%   (https://en.wikipedia.org/wiki/Peak\_signal-to-noise\_ratio).
% \end{thebibliography}

\end{document}