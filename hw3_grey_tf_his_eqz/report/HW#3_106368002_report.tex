%%%%%%%%%%%%%%%%%%%%%%%%%%%%%%%%%%%%%%%%%
% Academic Title Page
% LaTeX Template
% Version 2.0 (17/7/17)
%
% This template was downloaded from:
% http://www.LaTeXTemplates.com
%
% Original author:
% WikiBooks (LaTeX - Title Creation) with modifications by:
% Vel (vel@latextemplates.com)
%
% License:
% CC BY-NC-SA 3.0 (http://creativecommons.org/licenses/by-nc-sa/3.0/)
% 
% Instructions for using this template:
% This title page is capable of being compiled as is. This is not useful for 
% including it in another document. To do this, you have two options: 
%
% 1) Copy/paste everything between \begin{document} and \end{document} 
% starting at \begin{titlepage} and paste this into another LaTeX file where you 
% want your title page.
% OR
% 2) Remove everything outside the \begin{titlepage} and \end{titlepage}, rename
% this file and move it to the same directory as the LaTeX file you wish to add it to. 
% Then add \input{./<new filename>.tex} to your LaTeX file where you want your
% title page.
%
%%%%%%%%%%%%%%%%%%%%%%%%%%%%%%%%%%%%%%%%%

%----------------------------------------------------------------------------------------
%	PACKAGES AND OTHER DOCUMENT CONFIGURATIONS
%----------------------------------------------------------------------------------------

\documentclass[11pt]{article}

\usepackage[utf8]{inputenc} % Required for inputting international characters
\usepackage[T1]{fontenc} % Output font encoding for international characters

\usepackage{mathpazo} % Palatino font

\usepackage{comment} % enables the use of multi-line comments (\ifx \fi) 
\usepackage{lipsum} %This package just generates Lorem Ipsum filler text. 
% \usepackage{fullpage} % changes the margin
\usepackage{CJKutf8}
\usepackage{enumitem}
\usepackage{titlesec}
\usepackage[english]{babel}
\usepackage{blindtext}
\usepackage{graphicx}     % for figure
\usepackage{subcaption}   % for figure
\usepackage[export]{adjustbox}
\usepackage[most]{tcolorbox}
\usepackage{xcolor}

\usepackage{pgfplots}

\newenvironment{reportsection}[2]{%
\begin{list}{}{%
\setlength{\topsep}{0pt}%
\setlength{\leftmargin}{#1}%
\setlength{\rightmargin}{#2}%
\setlength{\listparindent}{\parindent}%
\setlength{\itemindent}{\parindent}%
\setlength{\parsep}{\parskip}%
}%
\item[]}{\end{list}}

\titlespacing*{\section}
{0pt}{0.5\baselineskip}{1\baselineskip}

\titlespacing*{\subsection}
{0pt}{0.1\baselineskip}{0.1\baselineskip}

\definecolor{codegreen}{rgb}{0,0.6,0}
\definecolor{codegray}{rgb}{0.5,0.5,0.5}
\definecolor{codepurple}{rgb}{0.58,0,0.82}
\definecolor{backcolour}{rgb}{0.95,0.95,0.92}

\lstdefinestyle{mystyle}{
    backgroundcolor=\color{backcolour},   
    commentstyle=\color{codegreen},
    keywordstyle=\color{magenta},
    numberstyle=\tiny\color{codegray},
    stringstyle=\color{codepurple},
    basicstyle=\footnotesize,
    breakatwhitespace=false,         
    breaklines=true,                 
    captionpos=b,                    
    keepspaces=true,                 
    numbers=left,                    
    numbersep=5pt,                  
    showspaces=false,                
    showstringspaces=false,
    showtabs=false,                  
    tabsize=2
}
\lstset{style=mystyle}

\begin{document}

%----------------------------------------------------------------------------------------
%	TITLE PAGE
%----------------------------------------------------------------------------------------

\begin{titlepage} % Suppresses displaying the page number on the title page and the subsequent page counts as page 1
	\newcommand{\HRule}{\rule{\linewidth}{0.5mm}} % Defines a new command for horizontal lines, change thickness here
						
	\center % Centre everything on the page
						
	%------------------------------------------------
	%	Headings
	%------------------------------------------------
						
	\textsc{\LARGE National Taipei University of Technology}\\[1.5cm] % Main heading such as the name of your university/college
						
	\textsc{\Large 2018 Fall}\\[0.5cm] % Major heading such as course name
						
	\textsc{\large 245765 - Advanced Digital Image Processing}\\[0.5cm] % Minor heading such as course title
						
	%------------------------------------------------
	%	Title
	%------------------------------------------------
						
	\HRule\\[0.4cm]
						
	{\huge\bfseries HW\#3 Grey Level Transformation \& Histogram Equalization}\\[0.4cm] % Title of your document
						
	\HRule\\[1.5cm]
						
	%------------------------------------------------
	%	Author(s)
	%------------------------------------------------
	\begin{CJK}{UTF8}{bsmi}
		\begin{minipage}{0.4\textwidth}
			\begin{flushleft}
				\large
				\textit{Author}\\
				106368002 張昌祺\\ 
				\textsc{Chang-Qi Zhang} \\
				justin840727@gmail.com % Your name
			\end{flushleft}
		\end{minipage}
		~
		\begin{minipage}{0.4\textwidth}
			\begin{flushright}
				\large
				\textit{Advisor}\\
				電子所 \\
				高立人 副教授 % Supervisor's name
			\end{flushright}
		\end{minipage}
	\end{CJK}
	% If you don't want a supervisor, uncomment the two lines below and comment the code above
	%{\large\textit{Author}}\\
	%John \textsc{Smith} % Your name
						
	%------------------------------------------------
	%	Date
	%------------------------------------------------
						
	\vfill\vfill\vfill % Position the date 3/4 down the remaining page
						
	{\large\today} % Date, change the \today to a set date if you want to be precise
						
	%------------------------------------------------
	%	Logo
	%------------------------------------------------
						
	\vfill
	\includegraphics[width=0.6\textwidth]{../../logo.jpg}\\[1cm] % Include a department/university logo - this will require the graphicx package
								
		%----------------------------------------------------------------------------------------
											
		% \vfill % Push the date up 1/4 of the remaining page

	\end{titlepage}
	\begin{reportsection}{-1cm}{-1cm}
		%----------------------------------------------------------------------------------------
		\section*{Problem 1~Grey Level Transformation (C/C++) (40\%)}
		\begin{enumerate}[label=\alph*.]
			\item 
      Enhance the image cat\_bright.raw and cat\_dark.raw by Power-Law and Piecewise-Linear 
      transformation that learned in class. Show the best parameters, the gray-level 
      transform curve and output images. (Figure, 20\%; Discussion, 10\%)
        % \begin{tikzpicture}
        %   % \includegraphics[width=\textwidth,height=4cm]{tiger}
        %   \begin{axis} [title={Power-Law Transformation},
        %       xlabel = {$x$},
        %       ylabel = {$y$},
        %       xmin = 0, 
        %       xmax = 255,
        %       ymin = 0, 
        %       ymax = 255,
        %     minor y tick num = 1]
        %     \addplot table [x=x, y=y0, col sep=comma, mark=none, smooth] {../result_plot_data/Power-Law.csv};
        %     \addplot table [x=x, y=y1, col sep=comma, mark=none, smooth] {../result_plot_data/Power-Law.csv};
        %     \addplot table [x=x, y=y2, col sep=comma, mark=none, smooth] {../result_plot_data/Power-Law.csv};
        %     \addplot table [x=x, y=y3, col sep=comma, mark=none, smooth] {../result_plot_data/Power-Law.csv};
        %     \addplot table [x=x, y=y4, col sep=comma, mark=none, smooth] {../result_plot_data/Power-Law.csv};
        %     \addplot table [x=x, y=y5, col sep=comma, mark=none, smooth] {../result_plot_data/Power-Law.csv};
        %     \addplot table [x=x, y=y6, col sep=comma, mark=none, smooth] {../result_plot_data/Power-Law.csv};
        %     \addplot table [x=x, y=y7, col sep=comma, mark=none, smooth] {../result_plot_data/Power-Law.csv};
        %     \addplot table [x=x, y=y8, col sep=comma, mark=none, smooth] {../result_plot_data/Power-Law.csv};
        %     \addplot table [x=x, y=y9, col sep=comma, mark=none, smooth] {../result_plot_data/Power-Law.csv};
        %     \addplot table [x=x, y=y10, col sep=comma, mark=none, smooth] {../result_plot_data/Power-Law.csv};
        %   \end{axis}
        % \end{tikzpicture}
    \subsection*{Ans}
		\textbf{Piecewise-Linear transformation}
    \begin{figure}[h]
      \centering
        \includegraphics[width=\textwidth]{{"./img_src/power_law"}.png}
      \caption{Power-Law Transformation in different Gamma.}
      \label{fig:power-law-curve}
    \end{figure}
    \newpage
		\begin{figure}[htbp]
			\centering
			\begin{subfigure}[b!]{0.43\linewidth}
			  \includegraphics[width=\textwidth]{{"../result_img/problem1/cat_bright_src"}.png}
			  \caption{Cat bright image source.}
			\end{subfigure}
			\hspace{1em}
			\begin{subfigure}[b!]{0.43\linewidth}
			  \includegraphics[width=\textwidth]{{"../result_img/problem1/power_law/cat_bright10.0"}.png} 
			  \caption{Power-Law Transformation.}
			\end{subfigure}
			\caption{Power-Law Transformation bright image with best \textbf{Gamma 10.0}.}
			\label{fig:power-law-curve}
    \end{figure}

    \begin{figure}[htbp]
			\centering
			\begin{subfigure}[b!]{0.43\linewidth}
			  \includegraphics[width=\textwidth]{{"../result_img/problem1/cat_dark_src"}.png}
			  \caption{Cat dark image source.}
			\end{subfigure}
			\hspace{1em}
			\begin{subfigure}[b!]{0.43\linewidth}
			  \includegraphics[width=\textwidth]{{"../result_img/problem1/power_law/cat_dark0.20"}.png} 
			  \caption{Power-Law Transformation.}
			\end{subfigure}
			\caption{Power-Law Transformation bright image with best \textbf{Gamma 0.20}.}
      \label{fig:row-col-replication-Lena}
    \end{figure}
    
    \newpage
    \textbf{Piecewise-Linear transformation}

    \begin{figure}[htbp]
      \centering
        \includegraphics[width=0.8\textwidth]{{"./img_src/PieceLinBright"}.png}
      \caption{Power-Law Transformation in different Gamma.}
      \label{fig:piec-b-curve}
    \end{figure}

    \begin{figure}[htbp]
			\centering
			\begin{subfigure}[b!]{0.43\linewidth}
			  \includegraphics[width=\textwidth]{{"../result_img/problem1/cat_bright_src"}.png}
			  \caption{Cat bright image source.}
			\end{subfigure}
			\hspace{1em}
			\begin{subfigure}[b!]{0.43\linewidth}
			  \includegraphics[width=\textwidth]{{"../result_img/problem1/cat_bright_plt"}.png} 
			  \caption{Transformed image.}
			\end{subfigure}
      \caption{Piecewise-Linear transformation bright image with \textbf
      {Figure~\ref{fig:piec-b-curve} curve}.}
      \label{fig:row-col-replication-Lena}
    \end{figure}

    \begin{figure}[htbp]
      \centering
        \includegraphics[width=0.8\textwidth]{{"./img_src/PieceLinDark"}.png}
      \caption{Power-Law Transformation in different Gamma.}
      \label{fig:piec-d-curve}
    \end{figure}
    \newpage
    \begin{figure}[htbp]
			\centering
			\begin{subfigure}[b!]{0.43\linewidth}
			  \includegraphics[width=\textwidth]{{"../result_img/problem1/cat_dark_src"}.png}
			  \caption{Cat dark image source.}
			\end{subfigure}
			\hspace{1em}
			\begin{subfigure}[b!]{0.43\linewidth}
			  \includegraphics[width=\textwidth]{{"../result_img/problem1/cat_dark_plt"}.png} 
			  \caption{Transformed image.}
			\end{subfigure}
      \caption{Piecewise-Linear transformation dark image with \textbf
      {Figure~\ref{fig:piec-d-curve} curve}.}
      \label{fig:row-col-replication-Lena}
    \end{figure}

    \item 
    Compare and discuss the results obtained by the two methods and explain the 
    difference.( Discussion, 10\%)
    \subsection*{Ans}
    \end{enumerate}
    \subsection*{Source code for Problem 1}
    \textit{grey\_level\_transformation.hpp}
    \lstinputlisting[language=C++]{../include/grey_level_transformation.hpp}
    \textit{grey\_level\_transformation.cpp}
    \lstinputlisting[language=C++]{../src/grey_level_transformation.cpp}
    \newpage

    \section*{Problem 2~Histogram Equalization (C/C++) (60\%)}
    \begin{enumerate}[label=\alph*.]
			\item 
        Plot the histogram of livingroom\_bright.raw and livingroom\_dark.raw. 
        Discuss the difference among these histograms. (Figure, 10\%; Discussion, 10\%)
        \subsection*{Ans}
        \begin{figure}[htbp]
          \centering
          \begin{subfigure}[b!]{0.46\linewidth}
            \includegraphics[width=\textwidth]{{"./img_src/livingroom_bright_his"}.png}
            \caption{Living room bright.}
          \end{subfigure}
          \hspace{1em}
          \begin{subfigure}[b!]{0.46\linewidth}
            \includegraphics[width=\textwidth]{{"./img_src/livingroom_dark_his"}.png} 
            \caption{Living room dark.}
          \end{subfigure}
          \caption{Histofram of living room image}
          \label{fig:row-col-replication-Lena}
        \end{figure}
      \item  
        Perform histogram equalization on livingroom\_bright.raw and livingroom\_dark.raw. 
        Plot their histograms after equalization and compare the results, will the result
        be the same and why? (Figure, 10\%; Discussion, 10\%)
        \subsection*{Ans}
        \begin{figure}[htbp]
          \centering
          \begin{subfigure}[b!]{0.46\linewidth}
            \includegraphics[width=\textwidth]{{"./img_src/livingroom_bright_his_eq"}.png}
            \caption{Living room bright.}
          \end{subfigure}
          \hspace{1em}
          \begin{subfigure}[b!]{0.46\linewidth}
            \includegraphics[width=\textwidth]{{"./img_src/livingroom_dark_his_eq"}.png} 
            \caption{Living room dark.}
          \end{subfigure}
          \caption{Histofram of living room image}
          \label{fig:row-col-replication-Lena}
        \end{figure}
      \newpage
      \item  
        If you perform histogram equalization on cat\_bright.raw and cat\_dark.raw, 
        will the result look good? Show the output images and explain what causes this 
        situation and how to improve it. (Figure, 10\%, Discussion, 10\%)
        \subsection*{Ans}
        \begin{figure}[htbp]
          \centering
          \begin{subfigure}[b!]{0.4\linewidth}
            \includegraphics[width=\textwidth]{{"../result_img/problem2/livingroom_bright_src"}.png}
            \caption{Cat dark image source.}
          \end{subfigure}
          \hspace{1em}
          \begin{subfigure}[b!]{0.4\linewidth}
            \includegraphics[width=\textwidth]{{"../result_img/problem2/livingroom_eq_bright_src"}.png} 
            \caption{Transformed image.}
          \end{subfigure}
          \caption{Piecewise-Linear}
          \label{fig:row-col-replication-Lena}
        \end{figure}

        \begin{figure}[htbp]
          \centering
          \begin{subfigure}[b!]{0.4\linewidth}
            \includegraphics[width=\textwidth]{{"../result_img/problem2/livingroom_dark_src"}.png}
            \caption{Cat dark image source.}
          \end{subfigure}
          \hspace{1em}
          \begin{subfigure}[b!]{0.4\linewidth}
            \includegraphics[width=\textwidth]{{"../result_img/problem2/livingroom_eq_dark_src"}.png} 
            \caption{Transformed image.}
          \end{subfigure}
          \caption{Piecewise-Linear}
          \label{fig:row-col-replication-Lena}
        \end{figure}

    \end{enumerate}
    \subsection*{Source code for Problem 1}
    \textit{histogram\_equalization.hpp}
    \lstinputlisting[language=C++]{../include/histogram_equalization.hpp}
    \textit{histogram\_equalization.cpp}
    \lstinputlisting[language=C++]{../src/histogram_equalization.cpp}
	% 	\begin{thebibliography}{9}
	% 		\bibitem{MSE-wiki} Wikipedia. \emph{Mean squared error}[online]. \\
	% 		Available from World Wide Web: (https://en.wikipedia.org/wiki/Mean\_squared\_error).
	% 		\bibitem{PSNR-wiki} Wikipedia. \emph{Peak signal-to-noise ratio}[online].\\
	% 		Available from World Wide Web: \\
	% 		(https://en.wikipedia.org/wiki/Peak\_signal-to-noise\_ratio).
	% 	\end{thebibliography}
										
	\end{reportsection}
					
\end{document}
