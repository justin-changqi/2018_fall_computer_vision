%%%%%%%%%%%%%%%%%%%%%%%%%%%%%%%%%%%%%%%%%
% Academic Title Page
% LaTeX Template
% Version 2.0 (17/7/17)
%
% This template was downloaded from:
% http://www.LaTeXTemplates.com
%
% Original author:
% WikiBooks (LaTeX - Title Creation) with modifications by:
% Vel (vel@latextemplates.com)
%
% License:
% CC BY-NC-SA 3.0 (http://creativecommons.org/licenses/by-nc-sa/3.0/)
% 
% Instructions for using this template:
% This title page is capable of being compiled as is. This is not useful for 
% including it in another document. To do this, you have two options: 
%
% 1) Copy/paste everything between \begin{document} and \end{document} 
% starting at \begin{titlepage} and paste this into another LaTeX file where you 
% want your title page.
% OR
% 2) Remove everything outside the \begin{titlepage} and \end{titlepage}, rename
% this file and move it to the same directory as the LaTeX file you wish to add it to. 
% Then add \input{./<new filename>.tex} to your LaTeX file where you want your
% title page.
%
%%%%%%%%%%%%%%%%%%%%%%%%%%%%%%%%%%%%%%%%%

%----------------------------------------------------------------------------------------
%	PACKAGES AND OTHER DOCUMENT CONFIGURATIONS
%----------------------------------------------------------------------------------------

\documentclass[11pt]{article}

\usepackage[utf8]{inputenc} % Required for inputting international characters
\usepackage[T1]{fontenc} % Output font encoding for international characters

\usepackage{mathpazo} % Palatino font

\usepackage{comment} % enables the use of multi-line comments (\ifx \fi) 
\usepackage{lipsum} %This package just generates Lorem Ipsum filler text. 
% \usepackage{fullpage} % changes the margin
\usepackage{CJKutf8}
\usepackage{enumitem}
\usepackage{titlesec}
\usepackage[english]{babel}
\usepackage{blindtext}
\usepackage{graphicx}     % for figure
\usepackage{subcaption}   % for figure
\usepackage[export]{adjustbox}
\usepackage[most]{tcolorbox}
\usepackage{xcolor}

\newenvironment{reportsection}[2]{%
\begin{list}{}{%
\setlength{\topsep}{0pt}%
\setlength{\leftmargin}{#1}%
\setlength{\rightmargin}{#2}%
\setlength{\listparindent}{\parindent}%
\setlength{\itemindent}{\parindent}%
\setlength{\parsep}{\parskip}%
}%
\item[]}{\end{list}}

\titlespacing*{\section} 
{-7pt}{2.5ex plus 1ex minus .2ex}{1.3ex plus .2ex}

\titlespacing*{\subsection}
{0pt}{0.1\baselineskip}{0.1\baselineskip}

\begin{document}

%----------------------------------------------------------------------------------------
%	TITLE PAGE
%----------------------------------------------------------------------------------------

\begin{titlepage} % Suppresses displaying the page number on the title page and the subsequent page counts as page 1
	\newcommand{\HRule}{\rule{\linewidth}{0.5mm}} % Defines a new command for horizontal lines, change thickness here
		
	\center % Centre everything on the page
		
	%------------------------------------------------
	%	Headings
	%------------------------------------------------
		
	\textsc{\LARGE National Taipei University of Technology}\\[1.5cm] % Main heading such as the name of your university/college
		
	\textsc{\Large 2018 Fall}\\[0.5cm] % Major heading such as course name
		
	\textsc{\large 245765 - Advanced Digital Image Processing}\\[0.5cm] % Minor heading such as course title
		
	%------------------------------------------------
	%	Title
	%------------------------------------------------
		
	\HRule\\[0.4cm]
		
	{\huge\bfseries HW\#4-1 Local Moment \&\\ Image Smoothing}\\[0.4cm] % Title of your document
		
	\HRule\\[1.5cm]
		
	%------------------------------------------------
	%	Author(s)
	%------------------------------------------------
	\begin{CJK}{UTF8}{bsmi}
		\begin{minipage}{0.4\textwidth}
			\begin{flushleft}
				\large
				\textit{Author}\\
				106368002 張昌祺\\ 
				\textsc{Chang-Qi Zhang} \\
				justin840727@gmail.com % Your name
			\end{flushleft}
		\end{minipage}
		~
		\begin{minipage}{0.4\textwidth}
			\begin{flushright}
				\large
				\textit{Advisor}\\
				電子所 \\
				高立人 副教授 % Supervisor's name
			\end{flushright}
		\end{minipage}
	\end{CJK}
	% If you don't want a supervisor, uncomment the two lines below and comment the code above
	%{\large\textit{Author}}\\
	%John \textsc{Smith} % Your name
		
	%------------------------------------------------
	%	Date
	%------------------------------------------------
		
	\vfill\vfill\vfill % Position the date 3/4 down the remaining page
		
	{\large\today} % Date, change the \today to a set date if you want to be precise
		
	%------------------------------------------------
	%	Logo
	%------------------------------------------------
		
	\vfill
	\includegraphics[width=0.6\textwidth]{../../logo.jpg}\\[1cm] % Include a department/university logo - this will require the graphicx package
			 
		%----------------------------------------------------------------------------------------
			
		% \vfill % Push the date up 1/4 of the remaining page
			
	\end{titlepage}
	\begin{reportsection}{-1cm}{-1cm}
		%----------------------------------------------------------------------------------------
		\section*{Problem 1 Local Moment (40\%)}
		\begin{enumerate}[label=\alph*.]
			\item 
        Show the output images of local mean and local variance by using mask size 3x3. 
        You may want to enhance the output image to see details. Discuss how do you 
        process the image boundary problem. (Figure, 10\%; Discussion, 10\%)
				\subsection*{Ans}
				For shifting mask around an image, we have to deal with boundary problem. In this 
				homework, I add padding on the image which size is $pad_{cols}=(mask_{cols}-1)/2$
				ans $pad_{rows}=(mask_{rows}-1)/2$. Feeding the pads value with \textbf{mirror reflection} 
				method.
				\begin{figure}[h!]
					\centering
					\begin{subfigure}[t]{0.49\linewidth}
						\includegraphics[width=\textwidth]{{"../result_img/problem1/car"}.png}
						\caption{Car image source.}
					\end{subfigure}
					\hfill 
					\begin{subfigure}[t]{0.49\linewidth}
						\includegraphics[width=\textwidth]{{"../result_img/problem1/local_mean"}.png} 
						\caption{Local mean.}
					\end{subfigure}
					\vskip\baselineskip
					\begin{subfigure}[t]{0.49\linewidth}
						\includegraphics[width=\textwidth]{{"../result_img/problem1/local_variance"}.png}
						\caption{Local standard deviation.}
					\end{subfigure}
					\hfill
					\begin{subfigure}[t]{0.49\linewidth}
						\includegraphics[width=\textwidth]{{"../result_img/problem1/local_variance_eq"}.png} 
						\caption{ Local standard deviation after enhancement.}
					\end{subfigure}
					\caption{Local mean and standard deviation with 3x3 kernel result.}
					\label{fig:p1_result}
				\end{figure}
        \newpage
      \item 
        Perform local enhancement on car\_360x240.raw by designing your own method based 
        on the local moments from (a), show the output result. You may refer the method 
        used in the textbook. Discuss how you set the parameters. (Figure, 10\%; 
        Discussion, 10\%)
				\subsection*{Ans}
				For enhancing this image, I performed it by Equation~(\ref{eqt:enhc}) which refer to textbook.
				When local mask mean $m_{m}$ is less than full image mean $m_{G}$ times a const $k_{0}$ and 
				local standard deviation $\sigma_{m}$ is between full image standard deviation $\sigma_{G}$ times
				const $k_1$ and $k_2$, then we enhance pixel value $f(x, y)$ by const $E$.  
				\begin{equation}
					g(x, y) = 
					\begin{cases} 
						E \times f(x, y)	& \text{if } m_{m} \leq k_{0}m_{G} \text{ and } 
																k_{1}\sigma_{G} \leq \sigma_{m} \leq k_{2}\sigma_{G}\\
						f(x, y)      			& \text{if } others
					\end{cases}
					\label{eqt:enhc}
				\end{equation}
				\begin{figure}[h]
					\centering
					\includegraphics[width=.6\textwidth]{{"../result_img/problem1/local_enhancement"}.png}
					\caption{Result of local enhancement.}
					\label{fig:local-enhancement}
				\end{figure}
        % \blindtext[1]
		\end{enumerate}
	\end{reportsection}
	% \newpage
  \begin{reportsection}{-1cm}{-1cm}
    \section*{Problem 2 Image Smoothing (60\%)}
		\begin{enumerate}[label=\alph*.]
			\item 
				Try to extract all sheep from sheep\_615x368.raw by performing low-pass filtering
				(Blurring) and thresholding. The output image is a binary image containing the 
				white blobs of sheep and black background. Try to remove the noise caused by 
				the grass as much as you can. Explain how you design your LPF masks and 
				threshold. (Figure, 10\%; Discussion, 10\%)
				\subsection*{Ans}
				For smooth the image, I performed median filter with kernel size $11 \times 11$. Setting 
				threshold by 96\% the highest pixels value. The LPF and thresholding result show in 
				Figure~\ref{fig:filter-masking}.
				\begin{figure}[h]
					\centering
					\begin{subfigure}[t]{0.49\linewidth}
						\includegraphics[width=\textwidth]{{"../result_img/problem2/median_filter"}.png}
						\caption{Median Filter.}
					\end{subfigure}
					\hfill 
					\begin{subfigure}[t]{0.49\linewidth}
						\includegraphics[width=\textwidth]{{"../result_img/problem2/th_median_filter_th"}.png} 
						\caption{Threshold with 96\% the highest values.}
					\end{subfigure}
					\caption{Filtering and Masking result}
					\label{fig:filter-masking}
				\end{figure}
				% \blindtext[1]
      \item 
				Use the designed mask image\ LRcorner\_615x368.raw as the ROI mask and perform 
				logic operation on the results (a) to extract the sheep from the lower left 
				corner. (Figure, 10\%)
				\subsection*{Ans}
				\begin{figure}[htbp]
					\centering
					\includegraphics[width=0.8\textwidth]{{"../result_img/problem2/sheep_lr_mask"}.png}
					\caption{Filtering and Masking result}
					\label{fig:filter-masking}
				\end{figure}
        % \blindtext[1]
    \end{enumerate}
	\end{reportsection}
	
\end{document}
