%%%%%%%%%%%%%%%%%%%%%%%%%%%%%%%%%%%%%%%%%
% Academic Title Page
% LaTeX Template
% Version 2.0 (17/7/17)
%
% This template was downloaded from:
% http://www.LaTeXTemplates.com
%
% Original author:
% WikiBooks (LaTeX - Title Creation) with modifications by:
% Vel (vel@latextemplates.com)
%
% License:
% CC BY-NC-SA 3.0 (http://creativecommons.org/licenses/by-nc-sa/3.0/)
% 
% Instructions for using this template:
% This title page is capable of being compiled as is. This is not useful for 
% including it in another document. To do this, you have two options: 
%
% 1) Copy/paste everything between \begin{document} and \end{document} 
% starting at \begin{titlepage} and paste this into another LaTeX file where you 
% want your title page.
% OR
% 2) Remove everything outside the \begin{titlepage} and \end{titlepage}, rename
% this file and move it to the same directory as the LaTeX file you wish to add it to. 
% Then add \input{./<new filename>.tex} to your LaTeX file where you want your
% title page.
%
%%%%%%%%%%%%%%%%%%%%%%%%%%%%%%%%%%%%%%%%%

%----------------------------------------------------------------------------------------
%	PACKAGES AND OTHER DOCUMENT CONFIGURATIONS
%----------------------------------------------------------------------------------------

\documentclass[11pt]{article}

\usepackage[utf8]{inputenc} % Required for inputting international characters
\usepackage[T1]{fontenc} % Output font encoding for international characters

\usepackage{mathpazo} % Palatino font

\usepackage{comment} % enables the use of multi-line comments (\ifx \fi) 
\usepackage{lipsum} %This package just generates Lorem Ipsum filler text. 
% \usepackage{fullpage} % changes the margin
\usepackage{CJKutf8}
\usepackage{enumitem}
\usepackage{titlesec}
\usepackage[english]{babel}
\usepackage{blindtext}
\usepackage{graphicx}     % for figure
\usepackage{subcaption}   % for figure
\usepackage[export]{adjustbox}
\usepackage[most]{tcolorbox}
\usepackage{xcolor}

\newenvironment{reportsection}[2]{%
\begin{list}{}{%
\setlength{\topsep}{0pt}%
\setlength{\leftmargin}{#1}%
\setlength{\rightmargin}{#2}%
\setlength{\listparindent}{\parindent}%
\setlength{\itemindent}{\parindent}%
\setlength{\parsep}{\parskip}%
}%
\item[]}{\end{list}}

\titlespacing*{\section} {-7pt}{2.5ex plus 1ex minus .2ex}{1.3ex plus .2ex}

\titlespacing*{\subsection}
{0pt}{0.1\baselineskip}{0.1\baselineskip}

\begin{document}

%----------------------------------------------------------------------------------------
%	TITLE PAGE
%----------------------------------------------------------------------------------------

\begin{titlepage} % Suppresses displaying the page number on the title page and the subsequent page counts as page 1
	\newcommand{\HRule}{\rule{\linewidth}{0.5mm}} % Defines a new command for horizontal lines, change thickness here
		
	\center % Centre everything on the page
		
	%------------------------------------------------
	%	Headings
	%------------------------------------------------
		
	\textsc{\LARGE National Taipei University of Technology}\\[1.5cm] % Main heading such as the name of your university/college
		
	\textsc{\Large 2018 Fall}\\[0.5cm] % Major heading such as course name
		
	\textsc{\large 245765 - Advanced Digital Image Processing}\\[0.5cm] % Minor heading such as course title
		
	%------------------------------------------------
	%	Title
	%------------------------------------------------
		
	\HRule\\[0.4cm]
		
	{\huge\bfseries HW\#4-2 Laplacian and Sobel Filtering \& Image Sharping}\\[0.4cm] % Title of your document
		
	\HRule\\[1.5cm]
		
	%------------------------------------------------
	%	Author(s)
	%------------------------------------------------
	\begin{CJK}{UTF8}{bsmi}
		\begin{minipage}{0.4\textwidth}
			\begin{flushleft}
				\large
				\textit{Author}\\
				106368002 張昌祺\\ 
				\textsc{Chang-Qi Zhang} \\
				justin840727@gmail.com % Your name
			\end{flushleft}
		\end{minipage}
		~
		\begin{minipage}{0.4\textwidth}
			\begin{flushright}
				\large
				\textit{Advisor}\\
				電子所 \\
				高立人 副教授 % Supervisor's name
			\end{flushright}
		\end{minipage}
	\end{CJK}
	% If you don't want a supervisor, uncomment the two lines below and comment the code above
	%{\large\textit{Author}}\\
	%John \textsc{Smith} % Your name
		
	%------------------------------------------------
	%	Date
	%------------------------------------------------
		
	\vfill\vfill\vfill % Position the date 3/4 down the remaining page
		
	{\large\today} % Date, change the \today to a set date if you want to be precise
		
	%------------------------------------------------
	%	Logo
	%------------------------------------------------
		
	\vfill
	\includegraphics[width=0.6\textwidth]{../../logo.jpg}\\[1cm] % Include a department/university logo - this will require the graphicx package
			 
		%----------------------------------------------------------------------------------------
			
		% \vfill % Push the date up 1/4 of the remaining page
			
	\end{titlepage}
	\begin{reportsection}{-2cm}{-2cm}
		%----------------------------------------------------------------------------------------
		\section*{Problem 2 ~ Image Smoothing and Sharpening}
		\begin{enumerate}[label=\alph*., start=3]
			\item 
				Perform Laplacian filtering and Sobel filtering on lena512.raw and lena512\_noise.
				raw with the corresponding mask below respectively. Note that you should combine 
				both directional Sobel filtering into one image. Show the output result and 
				discuss the noise effects on both filters (1 st order and 2 nd order). How do 
				you process the image boundary? (Figure, 10\%; Discussion, 5\%)
				\vspace{1em}
				\subsection*{Ans}
				In Figure~\ref{fig:laplacian}, I applied the Laplacian on both Lena original 
				image and Lena noised image. The result shows Laplacian filter is not able to 
				get the good edge from noised image. To solve boundary problem, I add padding on the image which size is $pad_{cols}=(mask_{cols}-1)/2$
				ans $pad_{rows}=(mask_{rows}-1)/2$. Feeding the pads value with \textbf{mirror reflection} 
				method.
				\begin{figure}[h!]
					\centering
					\begin{subfigure}[t]{0.49\linewidth}
						\includegraphics[width=\textwidth]{{"../result_img/laplacian"}.png}
						\caption{Lena after Laplacian filter.}
					\end{subfigure}
					\hfill 
					\begin{subfigure}[t]{0.49\linewidth}
						\includegraphics[width=\textwidth]{{"../result_img/laplacian_noise"}.png} 
						\caption{Lena noised after Laplacian filter.}
					\end{subfigure}
					\caption{Laplacian filter result with 3x3 kernel.}
					\label{fig:laplacian}
				\end{figure}
				\vspace{1em}

				If I compute those images with Sebel filter, it will get better edge result than
				Laplacian filter. For noised image, you can see that there are some edges in the 
				image, but the noise is still strong. 
				\begin{figure}[h!]
					\centering
					\begin{subfigure}[t]{0.49\linewidth}
						\includegraphics[width=\textwidth]{{"../result_img/sobel"}.png}
						\caption{Lena after Sobel filter.}
					\end{subfigure}
					\hfill 
					\begin{subfigure}[t]{0.49\linewidth}
						\includegraphics[width=\textwidth]{{"../result_img/sobel_noise"}.png} 
						\caption{Lena noised after Sobel filter.}
					\end{subfigure}
					\caption{Sobel filter result with 3x3 kernel.}
					\label{fig:p1_result}
				\end{figure}
			% \vspace{1em}
			
			\newpage
			\item 
				Using the equation below, perform LPF and arithmetic subtraction operations on
				lena512.raw to get the sharpened image. (Figure, 10\%; Discussion 5\%)
				\begin{equation}
					Sharpened = Original-\textbf{c}\times (LPF~of~Original), where \textbf{ c } is 
					~a~constant.
					\label{eq:unparp-mask}
				\end{equation}
					\subsection*{Ans}
					Here I use Equation~\ref{eq:unparp-mask} to get unsharp mask with different 
					\textbf{c} value. As Figure~\ref{fig:unsharp-mask-img} shows that If we use 
					higher \textbf{c} value. We will get the darker unsharp mask.
					\begin{figure}[h!]
						\centering
						\begin{subfigure}[t]{0.4\linewidth}
							\includegraphics[width=\textwidth]{{"../result_img/sharped_img_03"}.png}
							\caption{$\textbf{c} = 0.3.$}
						\end{subfigure}
						\hfill 
						\begin{subfigure}[t]{0.4\linewidth}
							\includegraphics[width=\textwidth]{{"../result_img/sharped_img_05"}.png} 
							\caption{$\textbf{c} = 0.5.$}
						\end{subfigure}
					\end{figure}
					\begin{figure}[h!]
						\ContinuedFloat 
						\begin{subfigure}[t]{0.4\linewidth}
							\includegraphics[width=\textwidth]{{"../result_img/sharped_img_08"}.png}
							\caption{$\textbf{c} = 0.8.$}
						\end{subfigure}
						\hfill
						\begin{subfigure}[t]{0.4\linewidth}
							\includegraphics[width=\textwidth]{{"../result_img/sharped_img_1"}.png} 
							\caption{ $\textbf{c} = 1.0.$}
						\end{subfigure}
						\caption{Lena subtract LPF Lena result with difference \textbf{c} values.}
						\label{fig:unsharp-mask-img}
					\end{figure}
					\newpage
					In Figure~\ref{fig:unsharp-n-mask-img}, I applied arithmetic subtraction 
					operation on the noised image.
					\begin{figure}[h!]
						\centering
							\includegraphics[width=0.4\textwidth]{{"../result_img/sharped_n_img_03"}.png}
						\caption{Lena subtract LPF Lena Noised result with $\textbf{c} = 0.3.$}
						\label{fig:unsharp-n-mask-img}
					\end{figure}

					If we want to get sharpened image, we need to base on the 
					Equation\ref{eq:unparp-image}. The results of sharpened image show in the Figure
					~\ref{fig:unsharp-img} and Figure~\ref{fig:unsharp-n-img}.
					\begin{equation}
						Sharpened = Original+\textbf{c}\times (Original - Blurred),~where \textbf{ c } is 
						~a~constant.
						\label{eq:unparp-image}
					\end{equation}
					\begin{figure}[h!]
						\centering
						\begin{subfigure}[t]{0.45\linewidth}
							\includegraphics[width=\textwidth]{{"../result_img/sharped2_03"}.png}
							\caption{$\textbf{c} = 0.3.$}
						\end{subfigure}
						\hfill 
						\begin{subfigure}[t]{0.45\linewidth}
							\includegraphics[width=\textwidth]{{"../result_img/sharped2_05"}.png} 
							\caption{$\textbf{c} = 0.5.$}
						\end{subfigure}
					\end{figure}
					\begin{figure}[h!]
						\ContinuedFloat 
						\begin{subfigure}[t]{0.45\linewidth}
							\includegraphics[width=\textwidth]{{"../result_img/sharped2_08"}.png}
							\caption{$\textbf{c} = 0.8.$}
						\end{subfigure}
						\hfill
						\begin{subfigure}[t]{0.45\linewidth}
							\includegraphics[width=\textwidth]{{"../result_img/sharped2_1"}.png} 
							\caption{ $\textbf{c} = 1.0.$}
						\end{subfigure}
						\caption{Sharped image with different \textbf{c} values.}
						\label{fig:unsharp-img}
					\end{figure}
					\newpage
					\begin{figure}[h!]
						\centering
							\includegraphics[width=0.45\textwidth]{{"../result_img/sharped2_n_08"}.png}
						\caption{Sharped noised image result with $\textbf{c} = 0.8.$}
						\label{fig:unsharp-n-img}
					\end{figure}

		\end{enumerate}	
	\end{reportsection}
	
\end{document}
