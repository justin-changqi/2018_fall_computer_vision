%%%%%%%%%%%%%%%%%%%%%%%%%%%%%%%%%%%%%%%%%
% Academic Title Page
% LaTeX Template
% Version 2.0 (17/7/17)
%
% This template was downloaded from:
% http://www.LaTeXTemplates.com
%
% Original author:
% WikiBooks (LaTeX - Title Creation) with modifications by:
% Vel (vel@latextemplates.com)
%
% License:
% CC BY-NC-SA 3.0 (http://creativecommons.org/licenses/by-nc-sa/3.0/)
% 
% Instructions for using this template:
% This title page is capable of being compiled as is. This is not useful for 
% including it in another document. To do this, you have two options: 
%
% 1) Copy/paste everything between \begin{document} and \end{document} 
% starting at \begin{titlepage} and paste this into another LaTeX file where you 
% want your title page.
% OR
% 2) Remove everything outside the \begin{titlepage} and \end{titlepage}, rename
% this file and move it to the same directory as the LaTeX file you wish to add it to. 
% Then add \input{./<new filename>.tex} to your LaTeX file where you want your
% title page.
%
%%%%%%%%%%%%%%%%%%%%%%%%%%%%%%%%%%%%%%%%%

%----------------------------------------------------------------------------------------
%	PACKAGES AND OTHER DOCUMENT CONFIGURATIONS
%----------------------------------------------------------------------------------------

\documentclass[11pt]{article}

\usepackage[utf8]{inputenc} % Required for inputting international characters
\usepackage[T1]{fontenc} % Output font encoding for international characters

\usepackage{mathpazo} % Palatino font

\usepackage{comment} % enables the use of multi-line comments (\ifx \fi) 
\usepackage{lipsum} %This package just generates Lorem Ipsum filler text. 
% \usepackage{fullpage} % changes the margin
\usepackage{CJKutf8}
\usepackage{enumitem}
\usepackage{titlesec}
\usepackage[english]{babel}
\usepackage{blindtext}
\usepackage{graphicx}     % for figure
\usepackage{subcaption}   % for figure
\usepackage[export]{adjustbox}
\usepackage[most]{tcolorbox}
\usepackage{xcolor}

\usepackage{scrextend} % add margin

\newenvironment{reportsection}[2]{%
\begin{list}{}{%
% \addtolength{\topmargin}{-.875in}
\setlength{\topsep}{0pt}%
\setlength{\leftmargin}{#1}%
\setlength{\rightmargin}{#2}%
\setlength{\listparindent}{\parindent}%
\setlength{\itemindent}{\parindent}%
\setlength{\parsep}{\parskip}%
}%
\item[]}{\end{list}}

\titlespacing*{\section} {-7pt}{2.5ex plus 1ex minus .2ex}{1.3ex plus .2ex}

\titlespacing*{\subsection}
{0pt}{0.1\baselineskip}{0.1\baselineskip}

\begin{document}

%----------------------------------------------------------------------------------------
%	TITLE PAGE
%----------------------------------------------------------------------------------------

\begin{titlepage} % Suppresses displaying the page number on the title page and the subsequent page counts as page 1
	\newcommand{\HRule}{\rule{\linewidth}{0.5mm}} % Defines a new command for horizontal lines, change thickness here
		
	\center % Centre everything on the page
		
	%------------------------------------------------
	%	Headings
	%------------------------------------------------
		
	\textsc{\LARGE National Taipei University of Technology}\\[1.5cm] % Main heading such as the name of your university/college
		
	\textsc{\Large 2018 Fall}\\[0.5cm] % Major heading such as course name
		
	\textsc{\large 245765 - Advanced Digital Image Processing}\\[0.5cm] % Minor heading such as course title
		
	%------------------------------------------------
	%	Title
	%------------------------------------------------
		
	\HRule\\[0.4cm]
		
	{\huge\bfseries HW\#5 2D Discrete Fourier Transform (DFT)}\\[0.4cm] % Title of your document
		
	\HRule\\[1.5cm]
		
	%------------------------------------------------
	%	Author(s)
	%------------------------------------------------
	\begin{CJK}{UTF8}{bsmi}
		\begin{minipage}{0.4\textwidth}
			\begin{flushleft}
				\large
				\textit{Author}\\
				106368002 張昌祺\\ 
				\textsc{Chang-Qi Zhang} \\
				justin840727@gmail.com % Your name
			\end{flushleft}
		\end{minipage}
		~
		\begin{minipage}{0.4\textwidth}
			\begin{flushright}
				\large
				\textit{Advisor}\\
				電子所 \\
				高立人 副教授 % Supervisor's name
			\end{flushright}
		\end{minipage}
	\end{CJK}
	% If you don't want a supervisor, uncomment the two lines below and comment the code above
	%{\large\textit{Author}}\\
	%John \textsc{Smith} % Your name
		
	%------------------------------------------------
	%	Date
	%------------------------------------------------
		
	\vfill\vfill\vfill % Position the date 3/4 down the remaining page
		
	{\large\today} % Date, change the \today to a set date if you want to be precise
		
	%------------------------------------------------
	%	Logo
	%------------------------------------------------
		
	\vfill
	\includegraphics[width=0.6\textwidth]{../../logo.jpg}\\[1cm] % Include a department/university logo - this will require the graphicx package
			 
		%----------------------------------------------------------------------------------------
			
		% \vfill % Push the date up 1/4 of the remaining page
			
	\end{titlepage}
	\begin{reportsection}{-1cm}{0cm}
		%----------------------------------------------------------------------------------------
		\section*{Problem 1~DFT 2D (65\%) (C/C++)}
		\begin{enumerate}[label=\alph*.]
			\item 
				Write your own DFT algorithm (with origin shifted to center) to perform on 
				lena512.raw and lena512\_noise.raw. Show the output of magnitude spectra and 
				execution time of each image. Discuss the difference of each result image. 
				(Figure, 10\%; Discussion, 5\%)
				\subsection*{Ans}
				\begin{figure}[h!]
					\centering
					\begin{subfigure}[t]{0.45\linewidth}
						\includegraphics[width=\textwidth]{{"../result_img/lena_dft"}.png}
						\caption{Lena 512 DFT.}
						\label{fig:lena-dft-org}
					\end{subfigure}
					\hfill 
					\begin{subfigure}[t]{0.45\linewidth}
						\includegraphics[width=\textwidth]{{"../result_img/lena_noise_dft"}.png} 
						\caption{Lena 512 noised DFT.}
						\label{fig:lena-dft-noise}
					\end{subfigure}
					\caption{Lena and Lena noised DFT results}
					\label{fig:lena-dft}
				\end{figure}
				\begin{figure}[h!]
					\centering
					\includegraphics[width=0.8\textwidth]{{"./img_src/dft_terminal"}.png} 
					\caption{DFT execution time}
					\label{fig:lena-dft}
				\end{figure}
				\newpage
			\item
				Write your own IDFT algorithm and test on DFT output from (a). Show the result images.
				Check whether your output images can be exactly transformed back to the original images or
				not. Discuss if there is any difference between them. (Figure, 10\%; Discussion, 5\%)
				\subsection*{Ans}
				\begin{figure}[h!]
					\centering
					\begin{subfigure}[t]{0.45\linewidth}
						\includegraphics[width=\textwidth]{{"../result_img/lena_idft"}.png}
						\caption{Lena 512 IDFT from Figure~\ref{fig:lena-dft-org}.}
						\label{fig:lena-idft-org}
					\end{subfigure}
					\hfill 
					\begin{subfigure}[t]{0.45\linewidth}
						\includegraphics[width=\textwidth]{{"../result_img/lena_noise_idft"}.png} 
						\caption{Lena 512 noised IDFT from Figure~\ref{fig:lena-dft-noise}.}
						\label{fig:lena-idft-noise}
					\end{subfigure}
					\caption{Lena and Lena noised IDFT results.}
					\label{fig:lena-idft}
				\end{figure}
				\newpage
			\item
				Use the OpenCV built-in DFT and IDFT function to redo (a) \& (b). Discuss the difference in
				execution time and result images. (Figure, 10\%; Discussion, 5\%)
				\subsection*{Ans}
				\begin{figure}[h!]
					\centering
					\begin{subfigure}[t]{0.45\linewidth}
						\includegraphics[width=\textwidth]{{"../result_img/cv_lena_dft"}.png}
						\caption{Lena 512 DFT.}
						\label{fig:cv-lena-dft-org}
					\end{subfigure}
					\hfill 
					\begin{subfigure}[t]{0.45\linewidth}
						\includegraphics[width=\textwidth]{{"../result_img/cv_lena_noise_dft"}.png} 
						\caption{Lena 512 noised DFT.}
						\label{fig:lena-dft-noise}
					\end{subfigure}
					\caption{Lena and Lena noised DFT results}
					\label{fig:lena-dft}
				\end{figure}
				\begin{figure}[h!]
					\centering
					\includegraphics[width=0.8\textwidth]{{"./img_src/cv_dft_terminal"}.png} 
					\caption{DFT execution time}
					\label{fig:lena-dft}
				\end{figure}
				\newpage
			\item	
				Use spatial median filter and low pass (mean) filter with mask size 5 on lena512\_noise.raw
				respectively. Following that, perform the DFT on them. (can Show the result images. Discuss
				the difference between these magnitude spectra. (Figure, 10\%; Discussion, 10\%)
				\subsection*{Ans}
				\begin{figure}[h!]
					\centering
					\begin{subfigure}[t]{0.45\linewidth}
						\includegraphics[width=\textwidth]{{"../result_img/lena_noise_box_dft"}.png}
						\caption{Lena 512 Box filter DFT.}
						\label{fig:lena-box-dft}
					\end{subfigure}
					\hfill 
					\begin{subfigure}[t]{0.45\linewidth}
						\includegraphics[width=\textwidth]{{"../result_img/lena_noise_median_dft"}.png} 
						\caption{Lena 512 Median filter DFT.}
						\label{fig:lena-median-df}
					\end{subfigure}
					\caption{Lena noise LPF DFT results}
					\label{fig:lena-lpf-dft}
				\end{figure}
		\end{enumerate}
	\end{reportsection}
	\begin{reportsection}{-1cm}{0cm}
		\section*{Problem 2~Filtering in Frequency (35\%)}
		\begin{addmargin}[1em]{0em}
			\textit{This problem tries to remove the background dot noises of the fingerprint 
			image finger\_point.raw.}
		\end{addmargin}
		\begin{enumerate}[label=\alph*.]
			\item 
				In order to analyze the noise, you need to use DFT to transform finger\_point.raw 
				to frequency domain. Show the output of the magnitude spectra result. 
				(Figure, 5\%; Discussion, 5\%)
				\subsection*{Ans}
				\begin{figure}[h!]
					\centering
					\begin{subfigure}[t]{0.45\linewidth}
						\includegraphics[width=\textwidth]{{"../result_img/fingerpoint_src"}.png}
						\caption{Fingerpoint source image.}
						\label{fig:fingerpoint-src}
					\end{subfigure}
					\hfill 
					\begin{subfigure}[t]{0.45\linewidth}
						\includegraphics[width=\textwidth]{{"../result_img/fingerpoint_source_sp"}.png} 
						\caption{Fingerpoint DFT.}
						\label{fig:fingerpoint-sp}
					\end{subfigure}
					\caption{Finger Point}
					\label{fig:fingerpoint}
				\end{figure}
				\newpage
			\item 
				Design a filter in frequency domain to remove the frequency of the noises. How do you design
				it? Explain it. (Figure, 5\%; Discussion, 5\%)
				\subsection*{Ans}
				\begin{figure}[h!]
					\centering
					\includegraphics[width=0.4\textwidth]{{"../result_img/filter"}.png}  
					\caption{Filter}
					\label{fig:lena-dft}
				\end{figure}
				\newpage
			\item
				Show the output of the magnitude spectra result after removing the noise. 
				Perform IDFT on the frequency-domain output to obtain the noise removal 
				fingerprint result (Figure, 10\%; Discussion, 5\%)
				\subsection*{Ans}

		\end{enumerate}
	\end{reportsection}
\end{document}
