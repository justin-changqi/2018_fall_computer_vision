%%%%%%%%%%%%%%%%%%%%%%%%%%%%%%%%%%%%%%%%%
% Academic Title Page
% LaTeX Template
% Version 2.0 (17/7/17)
%
% This template was downloaded from:
% http://www.LaTeXTemplates.com
%
% Original author:
% WikiBooks (LaTeX - Title Creation) with modifications by:
% Vel (vel@latextemplates.com)
%
% License:
% CC BY-NC-SA 3.0 (http://creativecommons.org/licenses/by-nc-sa/3.0/)
% 
% Instructions for using this template:
% This title page is capable of being compiled as is. This is not useful for 
% including it in another document. To do this, you have two options: 
%
% 1) Copy/paste everything between \begin{document} and \end{document} 
% starting at \begin{titlepage} and paste this into another LaTeX file where you 
% want your title page.
% OR
% 2) Remove everything outside the \begin{titlepage} and \end{titlepage}, rename
% this file and move it to the same directory as the LaTeX file you wish to add it to. 
% Then add \input{./<new filename>.tex} to your LaTeX file where you want your
% title page.
%
%%%%%%%%%%%%%%%%%%%%%%%%%%%%%%%%%%%%%%%%%

%----------------------------------------------------------------------------------------
%	PACKAGES AND OTHER DOCUMENT CONFIGURATIONS
%----------------------------------------------------------------------------------------

\documentclass[12pt]{article}

\usepackage[utf8]{inputenc} % Required for inputting international characters
\usepackage[T1]{fontenc} % Output font encoding for international characters

\usepackage{mathpazo} % Palatino font

\usepackage{comment} % enables the use of multi-line comments (\ifx \fi) 
\usepackage{lipsum} %This package just generates Lorem Ipsum filler text. 
% \usepackage{fullpage} % changes the margin
\usepackage{CJKutf8}
\usepackage{enumitem}
\usepackage{titlesec}
\usepackage[english]{babel}
\usepackage{blindtext}
\usepackage{graphicx}     % for figure
\usepackage{subcaption}   % for figure
\usepackage[export]{adjustbox}
\usepackage[most]{tcolorbox}
\usepackage{xcolor}
\usepackage{geometry}

\newenvironment{reportsection}[1]{%
\section*{#1}%
\vspace{-1em}
\begin{enumerate}[label=\alph*.,labelindent=\parindent, leftmargin=0em]{%
% \setlength{\topmargin}{0cm}%
% \setlength{\topsep}{0pt}%
% \setlength{\leftmargin}{0cm}%
% \setlength{\rightmargin}{0cm}%
% \setlength{\listparindent}{\parindent}%
% \setlength{\itemindent}{\parindent}%
% \setlength{\parsep}{\parskip}%
}%
\item[]}{\end{enumerate}}

\titlespacing*{\section} {-1.5em}{2.5ex plus 1ex minus .2ex}{1.3ex plus .2ex}
\titleformat{\section}
{\normalfont\fontsize{18}{15}\bfseries}{\thesection}{1em}{}

\titlespacing*{\subsection}
{0pt}{0.1\baselineskip}{0.1\baselineskip}

\begin{document}

%----------------------------------------------------------------------------------------
%	TITLE PAGE
%----------------------------------------------------------------------------------------

\begin{titlepage} % Suppresses displaying the page number on the title page and the subsequent page counts as page 1
	\newcommand{\HRule}{\rule{\linewidth}{0.5mm}} % Defines a new command for horizontal lines, change thickness here
		
	\center % Centre everything on the page
		
	%------------------------------------------------
	%	Headings
	%------------------------------------------------
		
	\textsc{\LARGE National Taipei University of Technology}\\[1.5cm] % Main heading such as the name of your university/college
		
	\textsc{\Large 2018 Fall}\\[0.5cm] % Major heading such as course name
		
	\textsc{\large 245765 - Advanced Digital Image Processing}\\[0.5cm] % Minor heading such as course title
		
	%------------------------------------------------
	%	Title
	%------------------------------------------------
		
	\HRule\\[0.4cm]
		
	{\huge\bfseries HW\#6-1 Frequency Domain Filter \& Geometric Transformation}\\[0.4cm] % Title of your document
		
	\HRule\\[1.5cm]
		
	%------------------------------------------------
	%	Author(s)
	%------------------------------------------------
	\begin{CJK}{UTF8}{bsmi}
		\begin{minipage}{0.4\textwidth}
			\begin{flushleft}
				\large
				\textit{Author}\\
				106368002 張昌祺\\ 
				\textsc{Chang-Qi Zhang} \\
				justin840727@gmail.com % Your name
			\end{flushleft}
		\end{minipage}
		~
		\begin{minipage}{0.4\textwidth}
			\begin{flushright}
				\large
				\textit{Advisor}\\
				電子所 \\
				高立人 副教授 % Supervisor's name
			\end{flushright}
		\end{minipage}
	\end{CJK}
	% If you don't want a supervisor, uncomment the two lines below and comment the code above
	%{\large\textit{Author}}\\
	%John \textsc{Smith} % Your name
		
	%------------------------------------------------
	%	Date
	%------------------------------------------------
		
	\vfill\vfill\vfill % Position the date 3/4 down the remaining page
		
	{\large\today} % Date, change the \today to a set date if you want to be precise
		
	%------------------------------------------------
	%	Logo
	%------------------------------------------------
		
	\vfill
	\includegraphics[width=0.6\textwidth]{../../logo.jpg}\\[1cm] % Include a department/university logo - this will require the graphicx package
			 
		%----------------------------------------------------------------------------------------
			
		% \vfill % Push the date up 1/4 of the remaining page
	\end{titlepage}

	\newgeometry{left=0.8in,right=0.8in,top=1in,bottom=1in}
	\begin{reportsection}{Problem 1~~~Filter in Frequency domain (60\%)}
		\item 
		Using ideal LPF, Gaussian LPF, with $D_{0}=15,~50$ respectively to filter 
		fox\_380x284.raw in frequency domain. Show the result of magnitude spectrum and 
		the output image by using IDFT. Discuss the visual difference between each result 
		image. (Figure 10\%; Discussion 10\%)
		\subsection*{Ans}
		From results Figure~\ref{fig:ideal_lpf_15} to \ref{fig:gaussian_lpf_50} you can see 
		the big difference between Idea Low Pass Filter and Gaussian Low Pass Filter is 
		the \textbf{Ringing effect}. Where it only happened in Idea Low Pass Filter.  
		\begin{figure}[h!]
			\centering
			\begin{subfigure}[t]{0.4\linewidth}
				\includegraphics[width=\textwidth]{{"../result_img/1.a/ideal_lpf_15_sp"}.png}
				\caption{Magnitude spectrum after apply filter.}
				\label{fig:ideal_lpf_15_sp}
			\end{subfigure}
			\hfill 
			\begin{subfigure}[t]{0.4\linewidth}
				\includegraphics[width=\textwidth]{{"../result_img/1.a/ideal_lpf_15_idft"}.png} 
				\caption{IDFT result after apply filter.}
				\label{fig:ideal_lpf_15_idft}
			\end{subfigure}
			\caption{Results of fox image apply \textbf{idea} filter with $D_0=15$}
			\label{fig:ideal_lpf_15}
		\end{figure}
		\begin{figure}[h!]
			\centering
			\begin{subfigure}[t]{0.4\linewidth}
				\includegraphics[width=\textwidth]{{"../result_img/1.a/ideal_lpf_50_sp"}.png}
				\caption{Magnitude spectrum after apply filter.}
				\label{fig:ideal_lpf_50_sp}
			\end{subfigure}
			\hfill 
			\begin{subfigure}[t]{0.4\linewidth}
				\includegraphics[width=\textwidth]{{"../result_img/1.a/ideal_lpf_50_idft"}.png} 
				\caption{IDFT result after apply filter.}
				\label{fig:ideal_lpf_50_idft}
			\end{subfigure}
			\caption{Results of fox image apply \textbf{idea} filter with $D_0=50$}
			\label{fig:ideal_lpf_50}
		\end{figure}
		\newpage
		\begin{figure}[h!]
			\centering
			\begin{subfigure}[t]{0.4\linewidth}
				\includegraphics[width=\textwidth]{{"../result_img/1.a/gaussian_lpf_15_sp"}.png}
				\caption{Magnitude spectrum after apply filter.}
				\label{fig:gaussian_lpf_15_sp}
			\end{subfigure}
			\hfill 
			\begin{subfigure}[t]{0.4\linewidth}
				\includegraphics[width=\textwidth]{{"../result_img/1.a/gaussian_lpf_15_idft"}.png} 
				\caption{IDFT result after apply filter.}
				\label{fig:gaussian_lpf_15_idft}
			\end{subfigure}
			\caption{Results of fox image apply \textbf{Gaussian} filter with $D_0=15$}
			\label{fig:gaussian_lpf_15}
		\end{figure}
		\begin{figure}[h!]
			\centering
			\begin{subfigure}[t]{0.4\linewidth}
				\includegraphics[width=\textwidth]{{"../result_img/1.a/gaussian_lpf_50_sp"}.png}
				\caption{Magnitude spectrum after apply filter.}
				\label{fig:gaussianlpf_50_sp}
			\end{subfigure}
			\hfill 
			\begin{subfigure}[t]{0.4\linewidth}
				\includegraphics[width=\textwidth]{{"../result_img/1.a/gaussian_lpf_50_idft"}.png} 
				\caption{IDFT result after apply filter.}
				\label{fig:gaussian_lpf_50_idft}
			\end{subfigure}
			\caption{Results of fox image apply \textbf{Gaussian} filter with $D_0=50$}
			\label{fig:gaussian_lpf_50}
		\end{figure}
		\newpage
		\item 
		Using Butterworth LPF with order = 1, 2, 5 and $D_{0}=15,~50$ to filter fox\_380x284.raw in
		frequency domain. Show the result of magnitude spectrum and the output image by using
		IDFT. Discuss the visual difference between each result image. Compare and discuss the
		result with (a) (Figure 10\%; Discussion 10\%)
		\subsection*{Ans}
		Butterworth Low Pass Filter is a that we easy to control the scope of the filter by
		filter order $n$. So that we can see the ringing effect when n is high
		(Figure~\ref{fig:bw_15_n5}).
		\begin{figure}[h!]
			\centering
			\begin{subfigure}[t]{0.4\linewidth}
				\includegraphics[width=\textwidth]{{"../result_img/1.b/butterworth_lpf_15_n1_sp_filtered"}.png}
				\caption{Magnitude spectrum after apply filter.}
				\label{fig:bw_15_n1_sp}
			\end{subfigure}
			\hfill 
			\begin{subfigure}[t]{0.4\linewidth}
				\includegraphics[width=\textwidth]{{"../result_img/1.b/butterworth_lpf_15_n1_idft"}.png} 
				\caption{IDFT result after apply filter.}
				\label{fig:bw_15_n1_idft}
			\end{subfigure}
			\caption{Results of fox image apply \textbf{Butterworth} filter with $D_0=15, n=1$}
			\label{fig:bw_15_n1}
		\end{figure}
		\begin{figure}[h!]
			\centering
			\begin{subfigure}[t]{0.4\linewidth}
				\includegraphics[width=\textwidth]{{"../result_img/1.b/butterworth_lpf_15_n2_sp_filtered"}.png}
				\caption{Magnitude spectrum after apply filter.}
				\label{fig:bw_15_n2_sp}
			\end{subfigure}
			\hfill 
			\begin{subfigure}[t]{0.4\linewidth}
				\includegraphics[width=\textwidth]{{"../result_img/1.b/butterworth_lpf_15_n2_idft"}.png} 
				\caption{IDFT result after apply filter.}
				\label{fig:bw_15_n2_idft}
			\end{subfigure}
			\caption{Results of fox image apply \textbf{Butterworth} filter with $D_0=15, n=2$}
			\label{fig:bw_15_n2}
		\end{figure}
		\newpage
		\begin{figure}[h!]
			\centering
			\begin{subfigure}[t]{0.4\linewidth}
				\includegraphics[width=\textwidth]{{"../result_img/1.b/butterworth_lpf_15_n5_sp_filtered"}.png}
				\caption{Magnitude spectrum after apply filter.}
				\label{fig:bw_15_n5_sp}
			\end{subfigure}
			\hfill 
			\begin{subfigure}[t]{0.4\linewidth}
				\includegraphics[width=\textwidth]{{"../result_img/1.b/butterworth_lpf_15_n5_idft"}.png} 
				\caption{IDFT result after apply filter.}
				\label{fig:bw_15_n5_idft}
			\end{subfigure}
			\caption{Results of fox image apply \textbf{Butterworth} filter with $D_0=15, n=5$}
			\label{fig:bw_15_n5}
		\end{figure}
		
		\begin{figure}[h!]
			\centering
			\begin{subfigure}[t]{0.4\linewidth}
				\includegraphics[width=\textwidth]{{"../result_img/1.b/butterworth_lpf_50_n1_sp_filtered"}.png}
				\caption{Magnitude spectrum after apply filter.}
				\label{fig:bw_50_n1_sp}
			\end{subfigure}
			\hfill 
			\begin{subfigure}[t]{0.4\linewidth}
				\includegraphics[width=\textwidth]{{"../result_img/1.b/butterworth_lpf_50_n1_idft"}.png} 
				\caption{IDFT result after apply filter.}
				\label{fig:bw_50_n1_idft}
			\end{subfigure}
			\caption{Results of fox image apply \textbf{Butterworth} filter with $D_0=50, n=1$}
			\label{fig:bw_50_n1}
		\end{figure}
		\begin{figure}[h!]
			\centering
			\begin{subfigure}[t]{0.4\linewidth}
				\includegraphics[width=\textwidth]{{"../result_img/1.b/butterworth_lpf_50_n2_sp_filtered"}.png}
				\caption{Magnitude spectrum after apply filter.}
				\label{fig:bw_50_n2_sp}
			\end{subfigure}
			\hfill 
			\begin{subfigure}[t]{0.4\linewidth}
				\includegraphics[width=\textwidth]{{"../result_img/1.b/butterworth_lpf_50_n2_idft"}.png} 
				\caption{IDFT result after apply filter.}
				\label{fig:bw_50_n2_idft}
			\end{subfigure}
			\caption{Results of fox image apply \textbf{Butterworth} filter with $D_0=50, n=2$}
			\label{fig:bw_50_n2}
		\end{figure}
		\newpage
		\begin{figure}[h!]
			\centering
			\begin{subfigure}[t]{0.4\linewidth}
				\includegraphics[width=\textwidth]{{"../result_img/1.b/butterworth_lpf_50_n5_sp_filtered"}.png}
				\caption{Magnitude spectrum after apply filter.}
				\label{fig:bw_50_n5_sp}
			\end{subfigure}
			\hfill 
			\begin{subfigure}[t]{0.4\linewidth}
				\includegraphics[width=\textwidth]{{"../result_img/1.b/butterworth_lpf_50_n5_idft"}.png} 
				\caption{IDFT result after apply filter.}
				\label{fig:bw_50_n5_idft}
			\end{subfigure}
			\caption{Results of fox image apply \textbf{Butterworth} filter with $D_0=50, n=5$}
			\label{fig:bw_50_n5}
		\end{figure}
		\item 
		Perform homomorphic filtering and try to find the best parameters on
		wolf\_dark\_256x256.raw, Discuss the visual difference of the result image. 
		(Figure 10\%; Discussion 10\%)
		\subsection*{Ans}
		Figure~\ref{fig:homomorphic} shows the result of wolf\_dark\_256x256.raw perform 
		Homomorphic filter with parameters $\gamma_L=1,~\gamma_H=10,~c=2,~d_0=100$. After 
		apply filter the image became more sharper than origin.
		\begin{figure}[h!]
			\centering
			\begin{subfigure}[t]{0.4\linewidth}
				\includegraphics[width=\textwidth]{{"../result_img/wolf_src"}.png}
				\caption{wolf\_dark\_256x256.raw original image.}
				\label{fig:h_src}
			\end{subfigure}
			\hfill 
			\begin{subfigure}[t]{0.4\linewidth}
				\includegraphics[width=\textwidth]{{"../result_img/1.c/wolf_homomorphic_idft"}.png} 
				\caption{IDFT result after apply filter.}
				\label{fig:h_idft}
			\end{subfigure}
			\caption{Results of wolf image apply \textbf{Homomorphic} filter with $\gamma_L=1,~ 
			\gamma_H=10,~c=2,~d_0=100$}
			\label{fig:homomorphic}
		\end{figure}
	\end{reportsection}
	\newpage
	\begin{reportsection}{Problem 2~~~Geometric transformation (40\%)}
		\item 
		Perform Wiener Filter and Inverse Filter on lena256\_blur.raw and 
		lena256\_blur\_gau.raw. Show the output images and discuss the visual 
		difference between each result image. (Figure 10\%; Discussion 10\%)
		\subsection*{Ans}
		\blindtext[1]
	\end{reportsection}
	
\end{document}
