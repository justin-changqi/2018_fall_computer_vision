\documentclass[a4paper, 11pt]{article}
    \usepackage{comment} % enables the use of multi-line comments (\ifx \fi) 
    \usepackage{lipsum} %This package just generates Lorem Ipsum filler text. 
    \usepackage{fullpage} % changes the margin
    \usepackage{CJKutf8}
    \usepackage{enumitem}
    \usepackage{titlesec}
    \usepackage[english]{babel}
    \usepackage{blindtext}
    \usepackage{graphicx}     % for figure
    \usepackage{subcaption}   % for figure
    \usepackage[export]{adjustbox}
    \usepackage[most]{tcolorbox}
    \usepackage{xcolor}
\renewcommand{\theenumi}{\alph{enumi}}

\titlespacing*{\section}
  {0pt}{0.5\baselineskip}{1\baselineskip}

\titlespacing*{\subsection}
  {0pt}{0.1\baselineskip}{0.1\baselineskip}

\begin{document}
%Header-Make sure you update this information!!!!
\noindent
\begin{center}
  \large\textbf{2018 Fall Advance Digital Image Processing Homework \#2-1} \\
\end{center}
\begin{CJK}{UTF8}{bsmi}
\normalsize EE 245765 \hfill \textbf{106368002 張昌祺 Justin, Chang-Qi Zhang} \\
Advisor: 電子所 高立人 \hfill justin840727@gmail.com \\
\null\hfill Due Date: 13:00pm, Oct 2 2018 \\
\end{CJK}

\section*{Problem 1}
\begin{enumerate}[label=\alph*.]
  \item 
    \blindtext[1]
    \subsection*{Ans}
    \blindtext[1]
\end{enumerate}

\begin{thebibliography}{9}
  \bibitem{MSE-wiki} Wikipedia. \emph{Mean squared error}[online]. \\
  Available from World Wide Web: (https://en.wikipedia.org/wiki/Mean\_squared\_error).
  \bibitem{PSNR-wiki} Wikipedia. \emph{Peak signal-to-noise ratio}[online].\\
  Available from World Wide Web: \\
  (https://en.wikipedia.org/wiki/Peak\_signal-to-noise\_ratio).
\end{thebibliography}

\end{document}